\documentclass{article}
\usepackage[utf8]{inputenc}
\usepackage[spanish]{babel}
\usepackage{graphicx}
\usepackage{float}
\usepackage{pdfpages}
\usepackage{listings}
\graphicspath{{./img/}}


\definecolor{codegreen}{rgb}{0,0.6,0}
\definecolor{codegray}{rgb}{0.5,0.5,0.5}
\definecolor{codepurple}{rgb}{0.58,0,0.82}
\definecolor{backcolour}{rgb}{0.95,0.95,0.92}

\lstdefinestyle{mystyle}{
	backgroundcolor=\color{backcolour},   
	commentstyle=\color{codegreen},
	keywordstyle=\color{magenta},
	numberstyle=\tiny\color{codegray},
	stringstyle=\color{codepurple},
	basicstyle=\footnotesize,
	breakatwhitespace=false,         
	breaklines=true,                 
	captionpos=b,                    
	keepspaces=true,                 
	numbers=left,                    
	numbersep=5pt,                  
	showspaces=false,                
	showstringspaces=false,
	showtabs=false,                  
	tabsize=2
}
\lstset{style=mystyle}

\title{Práctica 3. Primera parte. Ingeniería de requisitos: Análisis y especificación de requisitos}
\author{Noelia Escalera Mejías\\
	\and Alejandro Menor Molinero\\
	\and Javier Núñez Suárez\\
	\and Adra Sánchez Ruiz\\
	\and Jesús Torres Sánchez}

\begin{document}
	\maketitle
	
	\section{Primera heurística}
	La primera heurística que usaremos es bastante sencilla: escogeremos una ciudad inicial y a partir de ahí seleccionaremos la ciudad más cercana a la última escogida (que no haya sido seleccionada previamente) hasta que no queden ciudades por añadir al circuito. Haremos varias ejecuciones, empezando cada vez de una ciudad distinta, y escogeremos la opción con menos coste.
	
	\begin{lstlisting}[caption=Pseudocódigo de la primera heurística]
	VecinosCercanos(distancias, n, resultado){
		completados;
		todas_las_ciudades;
		
		// Metemos los indices de las ciudades
		for (i=1; i<=n; i++)
			todas_las_ciudades.insert(i);
			
		// Iniciamos cada vez en una ciudad diferente
		for(i=1; i<=n; i++){
			candidatos = todas_las_ciudades;
			candidatos.erase(i);
			seleccionados.push_back(i);
			distancia = 0;
			
			//Creamos el circuito de la ciudad por la que empezamos
			while (!candidatos.empty()){
				actual = seleccionados.back();
				mas_cercano = *candidatos.begin();
				min = distancias[actual][mas_cercano];
				
				// Averiguamos la ciudad mas cercana
				for(c : candidatos){
					d = distancias[actual][c];
					if (d < min){
						mas_cercano = c;
						min = d;
					}
				}
				
				seleccionados.push_back(mas_cercano);
				distancia += min;
				candidatos.erase(mas_cercano);
			}
			distancia += (distancias[seleccionados.front()][seleccionados.back()]);
			
			completados[distancia] = seleccionados;
		}
		resultado = completados.begin()->second;
	}
	\end{lstlisting}
	\section{Segunda Heurística}
	La segunda heurística consiste en conseguir un recorrido parcial que contenga algunas ciudades y posteriormente añadir las ciudades restantes al recorrido.
	
	\begin{lstlisting}[caption=Pseudocódigo de la segunda heurística]
	
	\end{lstlisting}
\end{document}