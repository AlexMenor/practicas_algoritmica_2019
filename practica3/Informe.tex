\documentclass{article}
\usepackage[utf8]{inputenc}
\usepackage[spanish]{babel}
\usepackage{graphicx}
\usepackage{longtable}
\usepackage{float}
\usepackage{amsmath}
\usepackage{listings}
\graphicspath{{./img/}}
\usepackage{color}

\title{Práctica 3. Algoritmos voraces - Salas de conferencias}

\author{Noelia Escalera Mejías \\
	\and Alejandro Menor Molinero \\
	\and Javier Núñez Suárez \\
	\and Adra Sánchez Ruiz \\
	\and Jesús Torres Sánchez}

\begin{document}
	\maketitle
	
	\section{Demostración (falta pulirla)}
	En esta demostración debemos tener presente que la solución óptima tiene que abrir tantas aulas como conferencias se solapen a la vez (preguntar si esto hay que demostrarlo también por favor).
	\begin{itemize}
		
		\item Sea $c_1$ una conferencia
		\item Sea $c_2$ la conferencia inmediatamente posterior y compatible con $c_1$
		\item Con nuesro algoritmo $aula(c_1)=aula(c_2)$. Llamemos $A$ a la solución que logra esto.
		\item Supongamos que en la solución óptima $aula(c_1) \neq aula(c_2)$
		\item Entonces se podría dar $n_{aulas}=n_{conferencias\ solapadas\ a\ la\ vez}+1$. Esto es absurdo ya que contradecimos la condición para que sea solución óptima.
		\item Luego la solución $A$ es la óptima.
	\end{itemize}
	
	
\end{document}